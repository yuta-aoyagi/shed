\usetheme[right]{Berkeley}

\title{スライドの\mbox{サンプル}} % adjust line-breaking on sidebar

\author{青柳\ 悠太}

\institute{引数として \texttt{\string\institute} に渡す所属が\\ないんだよなあ}

\date{2024年8月8~13日}

\begin{document}

\begin{frame}
  \titlepage
\end{frame}

\begin{frame}
  \tableofcontents
\end{frame}

\section{はじめに}

\begin{frame}{はじめに}
  \begin{alertblock}{注意}
    本発表は個人として行うものであり、所属する会社の公式見解ではありません。
  \end{alertblock}
\end{frame}

\begin{frame}[fragile]
  \frametitle{本発表について}

  電子メールの引用は \texttt{verbatim} 環境にしたい:
\begin{verbatim}
2024年6月 dd 日 HH:MM (差出人) <nobody@example.jp>:
> (引用したり)
> (後略)
\end{verbatim}
  \pause

  \bigskip
  \begin{block}{概要}
    実際に作ったスライドから本題を取り除いて、 Beamer のサンプルとして公開できるものにしてみる。
  \end{block}
\end{frame}

\subsection{subsection}

\begin{frame}[fragile]
  \frametitle{subsection}
  和文の文字同士の\ 間\ に\ スペースを入れたければ、バックスラッシュ \verb!\! をつける

  \bigskip
  箇条書き
  \begin{itemize}
  \item 上の\ スペース\ の件は、 {pdf\LaTeX} で \verb!\usepackage[whole]{bxcjkjatype}! を用いて \verb!CJK*! 環境に入る場合は必須
    (つけないとスペースが入らない)
  \item platex と dvipdfmx の組み合わせでは不要だが、つけても害はないはず
    \begin{itemize}
    \item (実は冒頭の \texttt{\string\author} で \verb*!\ ! をすでに使っている)
    \item ついでに入れ子の箇条書きも試している
    \end{itemize}
  \item ここは入れ子の外
  \end{itemize}
\end{frame}

\section{本題}

\newcommand{\myframe}[2]{
  \framebox{
    \begin{minipage}{#1}#2\end{minipage}
  }
}

\begin{frame}{本題: 概観}
  \begin{center}
    \myframe{0.75\textwidth}{
      外の枠

      \bigskip
      \hspace{\stretch{1}}
      \myframe{0.6\textwidth}{
        中の枠

        \bigskip
        \myframe{0.7\textwidth}{さらに内側}

        \bigskip
        \myframe{0.7\textwidth}{
          もう一つ内側

          \bigskip
          \framebox{最内\hspace{5em}}
        }
      }
      \hspace{\stretch{1}}
      \framebox{隣の枠}
      \hspace{\stretch{1}}
    }
  \end{center}
\end{frame}

\subsection{本題の subsection}

\begin{frame}{本題の subsection}
  聴衆に問いかけて十秒くらい考えていただく?
  \pause

  \bigskip
  \[
  (\text{合計}) - (\text{引く数}) = (\text{差})
  \]
\end{frame}

\subsection{subsection もう一つ}

\begin{frame}{subsection もう一つ}
  \begin{quote}
    引用文
  \end{quote}
  (引用元を明記、 URL とかもね)
\end{frame}

\section{付録}

\subsection{付録(1)}

\begin{frame}{付録(1)}
  書誌情報を2冊分。

  もう1 frame では Web ページを2つ紹介した。
\end{frame}

\end{document}
